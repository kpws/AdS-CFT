\documentclass[12pt]{article}

\usepackage[utf8]{inputenc}
\usepackage{amsmath}
%\usepackage{ae}
\usepackage{graphicx}
\usepackage{color}
\usepackage{tikz}
\usepackage[tc]{titlepic}

%\usepackage{bbm}
%\usepackage[swedish]{babel}
\newcommand{\N}{\ensuremath{\mathbbm{N}}}
\newcommand{\Z}{\ensuremath{\mathbbm{Z}}}
\newcommand{\Q}{\ensuremath{\mathbbm{Q}}}
\newcommand{\R}{\ensuremath{\mathbbm{R}}}
\newcommand{\C}{\ensuremath{\mathbbm{C}}}
\renewcommand{\d}{\ensuremath{\mathrm{d}}}
\newcommand{\e}{\ensuremath{\mathrm{e}}}
\renewcommand{\L}{\ensuremath{\mathcal{L}}}
\renewcommand{\i}{\ensuremath{i}}
%\renewcommand{\i}{\ensuremath{\mathrm{i}}}
\newcommand{\ket}[1]{|#1\rangle}
\newcommand{\bra}[1]{\langle#1|}
\newcommand{\braket}[2]{\bra{#1}#2\rangle}
\newcommand{\bracket}[3]{\bra{#1}#2\ket{#3}}
\newcommand{\fig}[3]{
\begin{figure}
\centering
\includegraphics{figs/#1}
\caption{#2}
\end{figure}
}
\title{Boundary Behaviour of Differential Equation for AdS/CFT Superconductivity}
\author{Petter Säterskog}

\begin{document}
\maketitle
The differential equation of interest
\begin{equation}
 \begin{split}
  (-2z^3+z^2\phi^2+2)\psi+(z^7+z^4-2z)\psi^\prime+(z^8-2z^5+z^2)\psi^{\prime\prime}&=0\\
  (z^5-z^2)\phi^{\prime\prime}+2\phi\psi^2&=0\\
 \end{split}
\end{equation}
or
\begin{equation}
 \begin{split}
  (-2z^3+z^2\phi^2+2)\psi+z\left((z^3-1)^2+3(z^3-1)\right)\psi^\prime+z^2(z^3-1)^2\psi^{\prime\prime}&=0\\
  z^2(z^3-1)\phi^{\prime\prime}+2\phi\psi^2&=0\\
 \end{split}
\end{equation}
The expansions of the solution at $z=0$ given boundary conditions $\mu$, $\rho$, $\psi_2$ and that $\psi^\prime(0)=0$ is
\begin{equation}
 \begin{split}
  \psi(z)&=\psi_2z^2-4\mu^2\psi_2\frac{z^4}{4!}+20\psi_2(2+\mu\rho)\frac{z^5}{5!}+ \mathcal{O}(z^6)\\
\phi(z)&=\mu-\rho z+4\mu\psi_2^2\frac{z^4}{4!}-12\rho\psi_2^2\frac{z^5}{5!} +\mathcal{O}(z^6)
 \end{split}
\end{equation}
It has numerically been found that starting at finite $z$ and integrating down towards $z=0$ with starting values such that $\psi^\prime(0)=0$ and $\rho$ is big gives $\rho/\psi_2=\sqrt{2}$.
\end{document}